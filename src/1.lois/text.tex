\section{Introduction}

L’action de l’homme sur l’environnement commence il y a environ 10.000 BP càd 10
000 ans avant 1950. Aujourd’hui l’homme est presque partout sur la terre où son
activité est perceptible partout.

Il commence il y a 10000 ans. On s’aperçoit que l’impact de l’homme sur
l’environnement ne commence qu’à partir de 1630 - au début de la révolution
industrielle- et explose littéralement au XX.

Ce sont plusieurs courbes décroissantes des derniers phénomènes dus à l’action
de l’homme sur l’environnement comme :
- la déforestation
- la consommation d’eau
- le taux de gaz carbonique dans l’air
- le nombre d’êtres humains
On peut distinguer deux types de phénomènes :
- Lents : par exemple la déforestation qui commence réellement vers 1650
mais qui n’a évoluée que d’un facteur 4 jusqu’aujourd’hui. La déforestation
entreprise par l’homme est légale en Europe.
- Rapides : par exemple la consommation d’eau .L’humanité actuellement
consomme 3600 km cube d’eau par ans (ce qui est égal à la capacité du lac
Huron), alors qu’en 1650 elle était de 100 km cube par ans.
L’évolution du nombre d’êtres humains a augmenté d’1/4 entre 1850 et 1950 et
doublé entre 1950 et aujourd’hui. Devant le constat de ces phénomènes, la
communauté scientifique a depuis longtemps réagit mais ce n’est que depuis peu
que le grand public et les médias sont réellement sensibilisés au problème.
La première conférence s’intéressant aux problèmes liés à la société et
l’environnement n’a eu lieu qu’en 1922 à Rio.