\subsection{Distributions échantillonnées}
\subsubsection{Distribution échantillonnées de la moyenne $\bar{X}$}
Supposons que la distribution de la population ait une moyenne $\mu$ et un écart-type $\sigma$. Il en est donc de même pour les distributions échantillonnées de $X_1, X_2,\ \dots\ , X_n$ et il résulte alors des propriétés de l’espérance mathématique que :
\begin{center}
$\begin{array}{LCL}
E(\bar{X}) &=& E\left(\dfrac{1}{n} \sum^n_{i=1} X_i\right)\\
&& \dfrac{1}{n} \sum^n_{i=1} E(X_i)\\
&& \dfrac{1}{n} \sum^n_{i=1} \mu\\
&& \mu
\end{array}$
\end{center}
et il résulte des propriétés de la variance que :
\begin{center}
	$\begin{array}{LCL}
		D^2(\bar{X}) &=& D^2\left(\dfrac{1}{n} \sum^n_{i=1} X_i\right)\\[0.5cm]
		\multicolumn{3}{L}{\boxed{D^2(aV+b) = a^2D^2(V)\hspace{0.5cm}\text{ \ref{distribution-variance} page \pageref{distribution-variance}}}}\\[0.3cm]
		&=& \dfrac{1}{n^2} \sum^n_{i=1}D^2(X_i)\\
		&=& \dfrac{1}{n^2} \sum^n_{i=1} \sigma\\
		&=& \dfrac{\sigma^2}{n}\\
	\end{array}$
\end{center}








\subsubsection{Distribution échantillonnées de la variance $S^2$}









\newpage
\subsubsection{Distribution échantillonnées d'une fonction de $\bar{X}$ et $s^2$}
Il résulte de ce qui précède que lorsque la population a une distribution normale, la variable
$$\boxed{\sqrt{n-1}\dfrac{\bar{X}-\mu}{S} \sim t_{(n-1)}}$$
\paragraph{Démonstration} La définition d'une chi carré $\chi^2_{(n-1)}$ est une somme de normale au carré $\displaystyle\sum_{i=1}^{n-1} (N_i(0,1))^2$.\\Or on a vu que $\chi^2_{(n-1)} = \dfrac{nS^2}{\sigma^2}$, donc
\begin{center}
	$\begin{array}{LCL}
		\chi^2_{(n-1)} &=& \sum_{i=1}^{n-1} (N_i(0,1))^2\\
		\dfrac{nS^2}{\sigma^2} &=& \sum_{i=1}^{n-1} \left(\dfrac{N_i(0,\sigma)}{\sigma}\right)^2\\
		\dfrac{nS^2}{\sigma^2} &=& \sum_{i=1}^{n-1} \dfrac{(N_i(0,\sigma))^2}{\sigma^2}\\
		nS^2 &=& \sum_{i=1}^{n-1} (N_i(0,\sigma))^2\\
		nS^2 \dfrac{1}{n-1}&=& \dfrac{1}{n-1} \sum_{i=1}^{n-1} (N_i(0,\sigma))^2\\
		\left(\dfrac{nS^2}{n-1}\right)^{\frac{-1}{2}} &=& \left(\dfrac{1}{n-1} \sum_{i=1}^{n-1} (N_i(0,\sigma))^2 \right)^{\frac{-1}{2}}\\
		\dfrac{1}{\sqrt{\dfrac{nS^2}{n-1}}} &=& \dfrac{1}{\sqrt{\dfrac{1}{n-1} \displaystyle\sum_{i=1}^{n-1} (N_i(0,\sigma))^2}}\\
		\dfrac{N(0,\sigma)}{S\sqrt{\dfrac{n}{n-1}}} &=& \dfrac{N(0,\sigma)}{\sqrt{\dfrac{1}{n-1} \displaystyle\sum_{i=1}^{n-1} (N_i(0,\sigma))^2}}\\
		\dfrac{N(0,\sigma)}{S\sqrt{\dfrac{n}{n-1}}} &=& t_{(n-1)}\\
	\end{array}$
\end{center}
On obtient bien une student $t_{(n-1)}$ à $n-1$ degrés de liberté (\ref{propriete-student} page~\pageref{propriete-student}). On va donc remplacer la normale $N(0,\sigma)$ dans l'équation par :
\begin{center}
	$\left\{\begin{array}{LCLL}
		\bar{X} &\sim& N\left(\mu,\dfrac{\sigma}{\sqrt{n}}\right)\\
		\dfrac{\bar{X}-\mu}{\dfrac{1}{\sqrt{n}}} &\sim& N(0,\sigma)\\
		\sqrt{n}\left(\bar{X}-\mu\right) &\sim& N(0,\sigma)\\
	\end{array}\right.$
\end{center}
$$t_{(n-1)} = \dfrac{N(0,\sigma)}{S\sqrt{\dfrac{n}{n-1}}} = \dfrac{\sqrt{n}\left(\bar{X}-\mu\right)}{S\sqrt{\dfrac{n}{n-1}}} = \dfrac{\bar{X}-\mu}{S\dfrac{1}{\sqrt{n-1}}} = \sqrt{n-1}\dfrac{\bar{X}-\mu}{S}$$
$$\boxed{t_{(n-1)} = \sqrt{n-1}\dfrac{\bar{X}-\mu}{S}}$$